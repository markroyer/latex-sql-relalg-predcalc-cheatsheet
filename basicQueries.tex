\documentclass[landscape]{article}
\usepackage[paperheight=8.5in,paperwidth=11.0in,margin=0.25in,landscape]{geometry}
\usepackage{graphicx}
\usepackage{hyperref}
\usepackage{textpos}
\usepackage{pdflscape}
\usepackage{listings}
\usepackage{tabularx}
\usepackage{array}

\usepackage{textcomp}

\pagenumbering{gobble}

% Relational algebra symbols are originally from
% ftp://reports.stanford.edu/www/dbgroup_only/latex-macros.html The
% link seems to be broken now?
\newcommand{\select}{\mbox{\Large$\sigma$}}
\newcommand{\cross}{\times}
\newcommand{\intersect}{\cap}
\newcommand{\union}{\cup}
\newcommand{\join}{\mbox{$\Join$}}
\newcommand{\leftsemijoin}{\mbox{$\mathrel{\raise1pt\hbox{\vrule height5pt
depth0pt width0.6pt\hskip-1.5pt$>$\hskip -2.5pt$<$}}$}}
\newcommand{\rightsemijoin}{\mbox{$\mathrel{\raise1pt\hbox{\hskip-1.5pt$>$\hskip -2.5pt$<$\hskip -1.1pt\vrule height5pt
depth0pt width0.6pt}}$}}
\newcommand{\project}{\mbox{\Large$\pi$}}
\newcommand{\aggregatefn}{\mbox{\Large$G$}}


\lstset{
  language=SQL,
  mathescape,
  basicstyle=\ttfamily, 
  columns=fullflexible, 
  upquote,
  morekeywords={data,wrapper,library,language}
}


% Footnote without a marker suggested by Nelson Lago from
% https://tex.stackexchange.com/questions/30720/footnote-without-a-marker
\newcommand\extrafootertext[1]{%
  \bgroup%
  \renewcommand\thefootnote{\fnsymbol{footnote}}%
  \renewcommand\thempfootnote{\fnsymbol{mpfootnote}}%
  \footnotetext[0]{#1}%
  \egroup%
}

\begin{document}


\newcommand{\tempwidth}{5.5cm}

\bgroup
\def\arraystretch{2.5}% Expand Vertical Spacing between rows 
\noindent
\begin{tabularx}{\textwidth}{|c|c|c|X|X|}

\multicolumn{5}{c}{ \textbf{\Large Basic Database Queries}} \\


\multicolumn{1}{r}{ \textbf{Assume}} & 
\multicolumn{4}{l}{$R=\{(1,2,3), (4,5,6), (7,8,9)\}$, $S=\{(1,1,1), (2,2,2)\}$, and $T=\{(1,2,3), (4,5,6)\}$} \\

\hline
\multicolumn{1}{|c|}{Operation} &
\multicolumn{1}{c|}{First-order Logic} &
\multicolumn{1}{m{3cm}|}{\shortstack{Relational Algebra \\ (of Sets)}} &
\multicolumn{1}{c|}{SQL} &
\multicolumn{1}{c|}{Result} \\

%%%%%%%%%%%%%%%%%%%%%%%%%%%%%%%%%%%%%%%%%%%%%%%%%%%%%%%%%%%%%%%%%%%%%%%%%%%%%%%%
 \hline \hline
%%%%%%%%%%%%%%%%%%%%%%%%%%%%%%%%%%%%%%%%%%%%%%%%%%%%%%%%%%%%%%%%%%%%%%%%%%%%%%%%


Projection & $\{(x,y) \mid \exists z : (x,y,z) \in R \}$ & $\project_{x,y}R$ & 
\multicolumn{1}{m{\tempwidth}|}{
\begin{lstlisting}^^J
SELECT DISTINCT x,y^^J
FROM R^^J
\end{lstlisting} }
  & $\{(1,2), (4,5), (7,8)\}$ \\

%%%%%%%%%%%%%%%%%%%%%%%%%%%%%%%%%%%%%%%%%%%%%%%%%%%%%%%%%%%%%%%%%%%%%%%%%%%%%%%%
 \hline
%%%%%%%%%%%%%%%%%%%%%%%%%%%%%%%%%%%%%%%%%%%%%%%%%%%%%%%%%%%%%%%%%%%%%%%%%%%%%%%%

  Selection & $\{(x,y,z) \in R \mid x \leq 6 \}$ & $\select_{x\leq 6}R$ & 
\multicolumn{1}{m{\tempwidth}|}{
\begin{lstlisting}^^J
SELECT DISTINCT x,y,z^^J
FROM R^^J
WHERE x <= 6^^J
\end{lstlisting} }
& $\{ (1,2,3), (4,5,6) \}$ \\

%%%%%%%%%%%%%%%%%%%%%%%%%%%%%%%%%%%%%%%%%%%%%%%%%%%%%%%%%%%%%%%%%%%%%%%%%%%%%%%%
 \hline
%%%%%%%%%%%%%%%%%%%%%%%%%%%%%%%%%%%%%%%%%%%%%%%%%%%%%%%%%%%%%%%%%%%%%%%%%%%%%%%%

  Cross Product & $\{(x,y,z,a,b,c) \mid (x,y,z) \in R \wedge (a,b,c) \in S \}$ & $R \cross S$ & 
\multicolumn{1}{m{\tempwidth}|}{
\begin{lstlisting}^^J
SELECT DISTINCT x,y,z,a,b,c^^J
FROM R, S^^J
\end{lstlisting} } &
\multicolumn{1}{m{7cm}|}{$\{(1,2,3,1,1,1), (1,2,3,2,2,2), (4,5,6,1,1,1),$ 
$(4,5,6,2,2,2), (7,8,9,1,1,1), (7,8,9,2,2,2)\}$ }\\

%%%%%%%%%%%%%%%%%%%%%%%%%%%%%%%%%%%%%%%%%%%%%%%%%%%%%%%%%%%%%%%%%%%%%%%%%%%%%%%%
 \hline
%%%%%%%%%%%%%%%%%%%%%%%%%%%%%%%%%%%%%%%%%%%%%%%%%%%%%%%%%%%%%%%%%%%%%%%%%%%%%%%%

Union & $\{(x,y,z) \mid (x,y,z) \in R \vee (x,y,z) \in S \}$ & $R \union S$ & 
\multicolumn{1}{m{\tempwidth}|}{
\begin{lstlisting}^^J
SELECT * FROM R^^J
UNION^^J
SELECT * FROM S^^J
\end{lstlisting} }
& \multicolumn{1}{m{7cm}|}{$\{ (1,1,1), (1,2,3), (2,2,2),$ $(4,5,6), (7,8,9) \}$} \\

%%%%%%%%%%%%%%%%%%%%%%%%%%%%%%%%%%%%%%%%%%%%%%%%%%%%%%%%%%%%%%%%%%%%%%%%%%%%%%%%
 \hline
%%%%%%%%%%%%%%%%%%%%%%%%%%%%%%%%%%%%%%%%%%%%%%%%%%%%%%%%%%%%%%%%%%%%%%%%%%%%%%%%

Intersection & $\{(x,y,z) \mid (x,y,z) \in R \wedge (x,y,z)
\in T \}$ & $R \intersect T$ &
\multicolumn{1}{m{\tempwidth}|}{
\begin{lstlisting}^^J
SELECT * FROM R^^J
INTERSECT^^J
SELECT * FROM T^^J
\end{lstlisting} }
& $\{ (1,2,3), (4,5,6) \}$ \\

%%%%%%%%%%%%%%%%%%%%%%%%%%%%%%%%%%%%%%%%%%%%%%%%%%%%%%%%%%%%%%%%%%%%%%%%%%%%%%%%
 \hline
%%%%%%%%%%%%%%%%%%%%%%%%%%%%%%%%%%%%%%%%%%%%%%%%%%%%%%%%%%%%%%%%%%%%%%%%%%%%%%%%

Difference & $\{(x,y,z) \mid (x,y,z) \in R \wedge (x,y,z) \not\in T \}$ & $R - T$ & 
\multicolumn{1}{m{\tempwidth}|}{
\begin{lstlisting}^^J
SELECT * FROM R^^J
EXCEPT^^J
SELECT * FROM T^^J
\end{lstlisting} }
& $\{(7,8,9)\}$ \\

%%%%%%%%%%%%%%%%%%%%%%%%%%%%%%%%%%%%%%%%%%%%%%%%%%%%%%%%%%%%%%%%%%%%%%%%%%%%%%%%
 \hline
%%%%%%%%%%%%%%%%%%%%%%%%%%%%%%%%%%%%%%%%%%%%%%%%%%%%%%%%%%%%%%%%%%%%%%%%%%%%%%%%

\end{tabularx}
\egroup

\subsubsection*{Other Notes:}

The keyword \texttt{DISTINCT} is required in the SQL queries for
projection, selection, and cross product in order to match the
first-order logic and relational algebra queries.  Without the
\texttt{DISTINCT} keyword in the \texttt{SELECT} clause, the SQL
results are multisets (bags).  The reason multisets are are used,
instead of sets by default, is to try to avoid any decrease in
performance due to duplicate removal.

\extrafootertext{This document was
  created by Mark Royer. You can find the source code at
  \url{https://github.com/markroyer/latex-sql-relalg-predcalc-cheatsheet}.}


\end{document}

%%  LocalWords:  SQL multisets
